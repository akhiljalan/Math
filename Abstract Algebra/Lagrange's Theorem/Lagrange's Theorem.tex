\documentclass[letterpaper]{article} % or whatever
\usepackage[margin = 0.5in]{geometry}
\usepackage{listings}
\usepackage{amsmath}
\usepackage{amscd}
\usepackage [pdftex]{graphicx}
\usepackage{amssymb}
\usepackage{mathtools}
\usepackage{fancyvrb}
\usepackage{amsthm}
\usepackage{enumerate}
\usepackage{float}
\usepackage{relsize}
\usepackage{hyperref}
\usepackage{wrapfig}
\usepackage{scrextend}
\usepackage{cancel}
\usepackage{tikz-cd}
\usepackage{setspace}
\usepackage{wasysym}
\usepackage{fancyhdr}
\usepackage{ifthen}
\usepackage{blindtext}
\usepackage{tikz}
\usepackage[nottoc,numbib]{tocbibind}

\usepackage{graphicx}
\graphicspath{ {images/} }



\usepackage{amsfonts}
\usepackage{amsmath}
\usepackage{textcomp}
\usepackage{wasysym}

\newcommand{\norm}{\|}
\newcommand{\arrbegin}{\begin{eqnarray}}
\newcommand{\arrend}{\end{eqnarray}}
\newcommand{\summ}{\sum\abslimits_}
\newcommand{\barr}{\overline} 
\newcommand{\setz}{z = x + iy}
\newcommand{\setzangle}{z = re^{i\theta}}
\newcommand{\absl}{\left|}
\newcommand{\absr}{\right|}
\newcommand{\minsec}{\subsubsection*}
\newcommand{\integr}{\int\limits_}
\newcommand{\integrC}{\int\limits_{C}^{}}
\newcommand{\prob}{\subsection*}
\newcommand{\del}{\partial}
\newcommand{\RR}{\mathbb{R}}
\newcommand{\CC}{\mathbb{C}}
\newcommand{\NN}{\mathbb{N}}
\newcommand{\rarr}{\Rightarrow}


\begin{document}
\title{Lagrange's Theorem}
\author{Akhil Jalan}
\date{Feb 10, 2017}
\maketitle


\prob{Introduction and Definitions} 

\textbf{Claim}: Let $G$ be a group and $H \leq G$ a subgroup. Let $\absl G \absr = n$ and $\absl H \absr = m$ for some $m, n \in \\ZZ^{+}$. Then $m | n$. 

We will make use of cosets of $H$. A \textbf{left coset of H} $aH$ is $\{a * h: h \in H\}$. Similarly, a \textbf{right coset of H} $Ha$ is $\{h * a: h \in H\}$. 

\prob{Lemma 1} 

If $a, b \in G$ are distinct, then $aH = bH$ or $aH \cap bH = \emptyset$. 

Proof: Suppose that $aH \cap bH \neq \emptyset$. Then for some $h_i, h_j \in H$, we know $ah_i = bh_j$. This implies $a = bh_{j}h_{i}^{-1}$ and implies $b = ah_{i}h_{j}^{-1}$. Then $aH \subseteq bH$ since for any $ah_k \in aH$, we know $ah_k = (bh_{j}h_{i}^{-1})h_k = b(h_{j}(h_{i})^{-1}h_k) \in bH$. And by symmetry, $bH \subset aH$. So either $aH = bH$ or $aH \cap bH = \emptyset$. 

\prob{Lemma 2} 

For any $a \in G$ the coset $\absl aH \absr = \absl H \absr$. 

Proof: Consider $aH$ for some $a \in G$. By Lemma 1, set is either the same as $H$ itself (we compare $aH$ to $eH = H$) or totally disjoint from it. If it is the same as $H$ then it has the same size, namely $m$. If it is totally disjoint from $m$, then each element is of the form $ah_{i}$ for some $h_i \in H$. Then we have $ah_{1} = ah_{2} \rarr h_1 = h_2$ by left cancellation. So $h_1 \neq h_2 \rarr ah_1 \neq ah_2$ by contrapositive. So right multiplication by each element of $H$ yields a unique value, and $\absl aH \absr = \absl H \absr$.

\prob{Main Proof} 

If $G = \{g_1, ..., g_n\}$ then the cosets $g_{1}H, ..., g_{n}H$ are exhaustive of $G$. For any $g' \in G$ we know that at the least, $g' \in g'H$. 

By Lemma 1, the cosets are either equal or distinct. By Lemma 2, they are all of the same size. By our argument just above, they are exhaustive of $G$. Therefore, the left cosets of $H$ partition $G$, and there must be $\frac{n}{m}$ of them. Then $m | n$. 

\prob{Remark} 

This theorem is perhaps poorly stated, because most of the significance comes from the realization that membership in a left coset of $H$ (and by symmetric arguments, a right coset) is an equivalence relation on $G$ itself. These cosets form a very natural partition of the group. We might find it useful to make the following corollaries. 

\prob{Corollary 1} 

If the order of a group $G$ is a prime $p \in \NN$ then the only subgroups of $G$ are $\{e\}$ (the $\textbf{trivial subgroup}$) and $G$ itself. 

Proof: Suppose that $G$ is a group of prime order. Then no nontrivial proper subgroup of $G$ can exist, because it would have to have an order which divides $p$. 

\prob{Corollary 2} 

Membership in a coset $aH$ is an equivalence relation on $G$. 

Proof: Every element $g \in G$ exists in one coset, and only one coset, by the main proof. So clearly there is an equivalence relation of belonging to the same coset, which we can denote $\sim$. 

i. Reflexive: $a \sim a$ since $a \in gH \leftrightarrow a \in gH$.

ii. Symmetric: $a \sim b$ means "$a$ is in the same coset as $b$" so clearly $b$ is in the same coset as $a$. So $b \sim a$. 

iii. Transitive: If $a \sim b$ and $b \sim c$ then $a$ is in the same coset as $b$ which is in the same coset as $c$. So $a, c$ are in the same coset. We conclude $a \sim c$. 

\prob{Corollary 3}

The cosets $aH, Ha$ are subgroups of $G$ iff $a \in H$. 

Proof: We'll prove for $aH$, since it is entirely symmetric for both sides. 

$\Leftarrow$: Suppose that $a \in H$. Then since $ae = a \in aH$, this means by Lemma 1 that $aH = H$. $aH = H$ is a subgroup.

$\rarr$: Suppose that $aH$ is a subgroup. Then we know $ah_{i} = e$ for some $h_i \in H$. By uniqueness of inverses, $h_i = a^{-1}$. But if $a^{-1} \in H$, then so is $a$ by closure under inverses. 

\end{document}